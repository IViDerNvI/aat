\subsection{基本公式及导出公式}

\subsubsection{基本公式}

定义 B 为期末额,A 为期初额,r 为增长率,则有以下基本公式:

\begin{equation*}
	\begin{cases}
		r = \frac{B-A}{A} \\
		B = A(1+r)        \\
		A = \frac{B}{1+r} \\
	\end{cases}
\end{equation*}

\subsubsection{导出公式}
\paragraph{隔期增长率}
定义 B 为期末额,A 为期初额,$r_1,r_2$ 分别为期初-期中增长率和期中-期末增长率,则有总增长率$r$:
\[
	r = r_1 + r_2 + r_1r_2
\]

\paragraph{部分增长率}

将总额分成两部分,定义$B_1$ 为分1期末额,$A_1$ 为分1期初额,$r_1$ 为分1增长率,$B_2$ 为分2期末额,$A_2$ 为分2期初额,$r_2$ 分2增长率,$B$ 为总期末额,$A$ 为总期初额,$r$ 为总增长率。则有:

\[
	r = \frac{A_1}{A}r_1 + \frac{A_2}{A}r_2
\]

\paragraph{乘积增长率}

定义 $B_1,B_2$ 为期末额,$A_1,A_2$ 为期初额,$r_1,r_2$ 为增长率,则两个量的乘积的增长率为:

\[
	r = r_1 + r_2 + r_1r_2
\]

\paragraph{比值增长率}

定义 $B_1,B_2$ 为期末额,$A_1,A_2$ 为期初额,$r_1,r_2$ 为增长率,则两个量的比值的增长率为:

\[
	r = \frac{r_1-r_2}{1+r_2}
\]

\paragraph{期初比重和期末比重}

将总额分成两部分,定义$B_1$ 为分1期末额,$A_1$ 为分1期初额,$r_1$ 为分1增长率,$B$ 为总期末额,$A$ 为总期初额,$r$为总增长率(即从$\frac{A_1}{A}$增长到$\frac{B_1}{B}$)。则:
\[
	\frac{A_1}{A} \frac{1+r_1}{1+r} = \frac{B_1}{B}
\]

\paragraph{比重增长率}

将总额分成两部分,定义$B_1$ 为分1期末额,$A_1$ 为分1期初额,$r_1$ 为分1增长率,$B$ 为总期末额,$A$ 为总期初额,$r$为总增长率(即从$\frac{A_1}{A}$增长到$\frac{B_1}{B}$)。则部分占比的增长率为:
\[
	r^\prime = \frac{r_1-r}{1+r}
\]

\paragraph{比重差}
将总额分成两部分,定义$B_1$ 为分1期末额,$A_1$ 为分1期初额,$r_1$ 为分1增长率,$B$ 为总期末额,$A$ 为总期初额,$r$为总增长率(即从$\frac{A_1}{A}$增长到$\frac{B_1}{B}$)。则比重差为:
\[
	\frac{B_1}{B} - \frac{A_1}{A} = \frac{r_1 - r}{1+r}\frac{A_1}{A} = \frac{A_1}{B} (r_1-r)
\]

\paragraph{年均增长}

定义 $B$ 为期末额,$A$ 为期初额,$r$ 为年均增长率,$n$ 为年数,则有:
\[
	(1+r)^n = \frac{B}{A}
\]

\paragraph{基期比重}
给定量 $B, A$及其增长率 $a, b$,则基期比重为:
\[
	\frac{B}{A} = \frac{1+a}{1+b}
\]