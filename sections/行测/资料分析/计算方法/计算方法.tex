\subsection{计算方法}

\subsubsection{小分互换法}

\paragraph{使用场景} 某个大数除以一个整数,可以转化为一个大数乘以一个百分数.

\paragraph{案例} 例如: 求 $235 \div 7$, 就可以转化为 $235 \times 0.142857$, 可以初步计算为$23.5 + 9.2 + 0.46 \approx 33.16$

\paragraph{常见小分互换} 一些常见的小分互换可以帮助快速计算:

\begin{multicols}{2}
	\begin{enumerate}
		\item $\frac{1}{2} = 0.500$
		\item $\frac{1}{3} = 0.330$
		\item $\frac{1}{4} = 0.250$
		\item $\frac{1}{5} = 0.200$
		\item $\frac{1}{6} = 0.167$
		\item $\frac{1}{7} = 0.143$
		\item $\frac{1}{8} = 0.125$
		\item $\frac{1}{9} = 0.111$
		\item $\frac{1}{10} = 0.100$
		\item $\frac{1}{11} = 0.091$
		\item $\frac{1}{12} = 0.083$
		\item $\frac{1}{13} = 0.077$
		\item $\frac{1}{14} = 0.071$
		\item $\frac{1}{15} = 0.067$
		\item $\frac{1}{16} = 0.063$
		\item $\frac{1}{17} = 0.059$
		\item $\frac{1}{18} = 0.056$
		\item $\frac{1}{19} = 0.053$
	\end{enumerate}
\end{multicols}

\subsubsection{拆分法}

\paragraph{使用场景} 当一个数乘以一个百分数时, 将这个百分数凑到最近的便于计算的数字.

\paragraph{案例} 例如: 求 $235 \times 0.14$, 就可以转化为 $235 \times (0.1 + 0.04)$, 可以初步计算为$23.5 + 9.4 \approx 32.9$

\paragraph{常见拆分} 一些常见的拆分可以帮助快速计算:

\begin{multicols}{2}
	\begin{enumerate}
		\item $45\% = 50\% - 5\%$
		\item $55\% = 50\% + 5\%$
		\item $15\% = 10\% + 5\%$
		\item $60\% = 50\% + 10\%$
		\item $95\% = 1 - 5\%$
		\item $90\% = 1 - 10\%$
	\end{enumerate}
\end{multicols}

\subsubsection{幂次估计}

\paragraph{使用场景} 计算年均增长率时, 可以使用幂次估算法来方便计算。

\paragraph{常见估计} 一些常见的幂次估算可以帮助快速计算:
\begin{table}[htbp]
	\centering
	\caption{不同 $ x $ 值对应的 $ r $(百分数,保留一位小数)}
	\label{tab:r_values}
	\begin{tabular}{|c|c|c|c|c|}
		\hline
		$ x $ & $ n=2 $ (\%) & $ n=3 $ (\%) & $ n=4 $ (\%) & $ n=5 $ (\%) \\ \hline
		1.05  & 2.5          & 1.6          & 1.2          & 1.0          \\ \hline
		1.10  & 4.9          & 3.2          & 2.4          & 1.9          \\ \hline
		1.15  & 7.2          & 4.8          & 3.6          & 2.8          \\ \hline
		1.20  & 9.5          & 6.3          & 4.7          & 3.7          \\ \hline
		1.25  & 11.8         & 7.7          & 5.7          & 4.6          \\ \hline
		1.30  & 14.0         & 9.1          & 6.8          & 5.4          \\ \hline
		1.35  & 16.2         & 10.5         & 7.8          & 6.2          \\ \hline
		1.40  & 18.3         & 11.9         & 8.8          & 7.0          \\ \hline
		1.45  & 20.4         & 13.2         & 9.7          & 7.7          \\ \hline
		1.50  & 22.5         & 14.5         & 10.7         & 8.4          \\ \hline
		1.55  & 24.5         & 15.7         & 11.6         & 9.2          \\ \hline
		1.60  & 26.5         & 17.0         & 12.5         & 9.9          \\ \hline
		1.65  & 28.5         & 18.2         & 13.3         & 10.5         \\ \hline
		1.70  & 30.4         & 19.3         & 14.2         & 11.2         \\ \hline
		1.75  & 32.3         & 20.5         & 15.0         & 11.8         \\ \hline
		1.80  & 34.2         & 21.6         & 15.8         & 12.5         \\ \hline
		1.85  & 36.0         & 22.8         & 16.6         & 13.1         \\ \hline
		1.90  & 37.8         & 23.9         & 17.4         & 13.7         \\ \hline
		1.95  & 39.6         & 24.9         & 18.2         & 14.3         \\ \hline
		2.00  & 41.4         & 26.0         & 18.9         & 14.9         \\ \hline
	\end{tabular}
\end{table}
