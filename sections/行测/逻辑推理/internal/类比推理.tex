\subsection{类比推理}

\subsubsection{类比推理的基本关系}

\paragraph{全同关系} 二者是同一事物的不同称呼

\paragraph{包含关系} 包括组成关系和种属关系

\paragraph{并列关系} 二者是同类事物的不同内容,包括结构并列和含义并列

\paragraph{交叉关系} 二者是不同的分类方式,互有重叠

\paragraph{对立关系} 包括矛盾关系和反对关系(矛盾指两者完全互斥,非此即彼;反对指两者不相交,但并不一定互斥)

\paragraph{前后关系} 二者是事物发展的先后顺序

\paragraph{联系关系} 近年来的高频考点,主要包括:

\begin{itemize}
    \item 功能联系:包括主要功能和次要功能
    \item 原材料联系:事物的原材料或者材质
    \item 场所联系:作用的场所或发生的地点
    \item 时间联系:包括先后顺序等(尤其考虑时间联系的紧密程度,十二时辰中各个时辰间隔程度)
    \item 因果联系:二者是事物运动的因果联系
    \item 职业联系:二者是职业或行业的联系
\end{itemize}

\subsubsection{类比推理的常见考法}

见附录\ref{trd:appendix}。

