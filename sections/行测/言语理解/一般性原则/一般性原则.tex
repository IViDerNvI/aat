\subsection{一般性原则}

\paragraph{主体一致性原则}

主体一致原则是指, 文段和选项所讲述的主体要一致。比如文段讲的是“三农问题”,选项讲的是“智慧农业”,则不符合主体一致性原则。通常情况下, 主体会在文段中反复提及, 也会在总结句中首次表达, 需要厘清其中关系。

\begin{quote}
	\textbf{一定要注意文段''主要讲述''和''举例子''的区别:}
	\begin{tcolorbox}[colback=blue!5, colframe=blue!75!black, title=主体一致性案例1]
		\textit{原始体育的萌芽与日复一日地生产劳动分不开,跳绳运动也不例外。古时,跳绳所用的绳被称为“绳索”,它是由古人编结而成的,人们在编绳索的过程中,通常会有一些跨越的动作,这些下意识的行为吸引了活泼好动的孩子,他们就用短的绳子在旁边反复模仿,并逐渐摸索出一些简单的跨越绳子的方法,当成一种游戏来玩,于是跳绳这一活动就产生了。对于跳绳的明确起源众说纷纭。最早出现的史料是汉代画像石上的跳绳图,证明当时已有了跳绳活动。南朝《荆楚岁时记》中有“飞百索”的记载,正是后来的跳绳游戏。}
		\tcblower%ww
		\texttt{此段文章只在开头提到原始体育, 作为背景资料, 引出跳绳运动也与劳动分不开。文章主体是跳绳运动, 所以以“原始体育”为主体的不符合主体一致性原则。}
	\end{tcolorbox}
\end{quote}

\begin{quote}
	\textbf{内容相近一定要注意文段主体:}
	\begin{tcolorbox}[colback=blue!5, colframe=blue!75!black, title=主体一致原则案例2]
		\textit{成绩要求相对较低,导致部分考生视艺考为进入高等学府的“绿色通道”或迅速成名的捷径,于是扎堆报考、突击报考等现象也就随之出现。另外,由于高中教学与艺考不能有效衔接,一个巨大的艺考培训市场应运而生,但市场内部存在行业垄断、恶性竞争、干扰院校招考正常秩序等乱象,破坏了艺术类专业招生和培养的健康生态。对此,教育部门应针对时弊,对艺术类专业招生加强规范,严格入校管理,进一步完善优化招录程序。}\\
		这段文字意在说明:\\
		A.“艺考”不等于“易考”\\
		B.艺术类专业招生亟待规范\\
		C.艺术类人才更需提升文化素养\\
		D.教育部门应多举措促进艺考公平
		\tcblower%ww
		\texttt{关注 B, D 选项, 两者都提到规范作用,但是看到文段中强调主体为'艺术类专业招生', 故选择 B 选项}
	\end{tcolorbox}
\end{quote}

\paragraph{对策优先原则}

对策优先原则是指, 文段中如果有对策类的内容, 则选项中也要有对策类的内容。比如文段中提到“我们应该采取措施来解决问题”,则选项中也要有“我们应该采取措施”这样的表述。

\begin{quote}
	\textbf{文段只描述问题时也可以酌情选择对策:}
	\begin{tcolorbox}[colback=blue!5, colframe=blue!75!black, title=对策优先原则案例]
		\textit{人车争道是影响我国城市交通安全的顽疾之一。“斑马线之争”不仅让市民出行心惊胆战,造成交通拥堵事故频发,还严重影响了公共交通文明和城市形象。据公安部交管局统计,近三年来,全国共在斑马线发生机动车与行人的交通事故 1.4 万起,造成 3898 人死亡。}\\
		这段文字主要阐述的是:\\
		A.交通出行与每个人生活息息相关\\
		B.倡导礼让斑马线需要长期坚持\\
		C.斑马线文明是城市文明的缩影\\
		D.机动车驾驶员法律意识、文明意识淡薄
		\tcblower%
		\texttt{文段表述了交通问题的严重, 没有给出解决方案, 选项 B 中的'倡导礼让斑马线'看作对策是针对题中`斑马线问题`的回答, 故选择 B 选项}
	\end{tcolorbox}
\end{quote}
