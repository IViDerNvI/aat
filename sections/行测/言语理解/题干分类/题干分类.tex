\subsection{题干分类}

\subsubsection{下面讲述}

\paragraph{文段讲述问题} 下文讲述解决问题的方法或者问题出现的原因
\begin{quote}
	\textbf{解决问题和分析问题同时出现时优先分析问题}
	\begin{tcolorbox}[colback=blue!5, colframe=blue!75!black, title=文段讲述问题案例]
		\textit{冬季是心血管疾病的高发季节。相关研究数据表明,每年的 12 月至次年 3 月是心血管病发病高峰期,发病数明显高于其他月份,这与血脂季节性波动的结论也相一致。据统计数据显示,因心血管疾病死亡事件中约 15.8\%与天气寒冷相关,而仅有 1.3\%与天气炎热相关。心力衰竭、心肌梗死和脑卒中在寒冷季节的发病率和死亡率均高于温暖季节。}\\
		这段文字接下来最有可能讲的是:\\
		A.心血管疾病为何冬季高发\\
		B.心血管疾病的症状有哪些\\
		C.如何有效治疗心血管疾病\\
		D.如何预防冬季心血管疾病
		\tcblower%
		\texttt{A,D分别讲的是文段提出问题的分析和解决方式,同时出现时优先选择问题分析,故选A}
	\end{tcolorbox}
\end{quote}

\paragraph{文段提出新概念} 下文解释新概念
\paragraph{文段给出总对策} 下文具体展开内容
\paragraph{文段提出过去内容} 下文讲述当今的新特点
\begin{quote}
	\begin{tcolorbox}[colback=blue!5, colframe=blue!75!black, title=文段提出过去内容案例]
		\textit{以往我们的农民形象塑造形成两种习惯性模式:一种是正面讲述主人公如何在改革与保守或先进与落后等二元对立中凭借忍辱负重、自我牺牲或善良人品等方式化敌为友,赢得村民信赖,取得改革的进展;另一种则是通过主人公的一连串小品化或喜剧化故事去消解矛盾,保持对观众的吸引力。这两种习惯性模式的美学后果在于,农民形象几乎成了一成不变的符号,缺乏应有的丰富性和吸引力,与现实生活存在巨大落差。这种模式是对真实本身的扁平化。}\\
		这段文字接下来最可能说的是:\\
		C.现在塑造的农民形象有什么改变
		\tcblower%
		\texttt{注意首句'以往',末尾提到两个模式的问题,故下文应强调现在的变化}
	\end{tcolorbox}
\end{quote}

\subsubsection{最佳题目}

\paragraph{措施优先原则}

文段若是说理文、则选号召的措施

\paragraph{新闻选新原则}

文段若是新闻稿,则优先选择新闻的''新''

\subsubsection{填入语句}

\paragraph{填入语句在开头} 填入语句可能是总概括、引出话题(包含主题词)、转折引出观点或对策等。(国考用俗语引出概率较大)
\paragraph{填入语句在中间} 填入语句可能是承上启下、前对策(后文是论证)、提出问题(后文是对策)等。
\paragraph{填入语句在结尾} 填入语句多为总结或对策句,注意首尾呼应。
\paragraph{其他注意} 填入语句要和前后句关系正确,要注意语法结构正确(主谓宾搭配得当)、逻辑关联词运用得当、全
文主语一致、后文指示代词指代正确;可特别考虑句式一致。

\begin{quote}
	\begin{tcolorbox}[colback=blue!5, colframe=blue!75!black, title=考虑句式一致]
		\textit{讲好中国故事、传播好中国声音,习近平总书记为我们树立了光辉典范。引用当地谚语、结合个人经历、 讲述感人故事……党的十八大以来, 习近平总书记在出访、 出席重要国际会议、 会见来访外国客人时,用外界听得懂的语言,\underline{\hspace{2cm}}。}\\
		填入文中画横线部分最恰当的一项是:\\
		A.创新对外话语表达,展示真实、立体、全面的中国\\
		B.以润物细无声的方式,增进世界对中国的了解\\
		C.立足本国,面向世界,让中国文化不断传播出去\\
		D.让中国理念、中国主张、中国方案更具触发共鸣的力量\\
		\tcblower%
		\texttt{B选项中'以润物细无声的方式'与'用外界听得懂的语言'对仗,优先选择}
	\end{tcolorbox}
\end{quote}

\subsubsection{语句排序}

\paragraph{整体把握} 把握结构特点(总分、并列)
\paragraph{细节把握} 下定义是首句的概率较大,指代一定要有前文
\paragraph{逻辑关系} 因果、转折、递进等关系确定两个一定相邻的子句

\subsubsection{细节判断}

\paragraph{细节题错误设置} 偷换概念,混淆时态,杂糅,本身有误(和实际不符),绝对化,无中生有
