\subsection{选项分类}

\subsubsection{错误分类}

\paragraph{出处错误} 背景、例子、分析论证中提炼的选项出处有误,一般不选
\paragraph{片面}原文并列谈论两方面,只说一方面的不能选(选项为对策可包容此问题)
\paragraph{杂糅}选项为原文中各个句子中提炼词语拼凑而成
\paragraph{绝对化}最高级、必要条件,原文未提及时不能选
\paragraph{下定义}即中性表达,介绍说明,在说理类文段题目中不选
\paragraph{无中生有}选项中的关键信息原文未提及
\paragraph{本身有误}即选项本身不符合事实,逻辑错误或不符合主流价值观等

\subsubsection{表述分类}

\paragraph{原因类}
文段为因果或果因结构,或整个文段是“选项”的原因;
\begin{quote}
	\begin{tcolorbox}[colback=blue!5, colframe=blue!75!black, title=原因类选项案例]
		\textit{员工家庭福利政策的长期缺失,虽然给企业带来了成本降低、产量提高等显而易见的优势,但对员工而言,抛弃年幼的孩子和年迈的父母,显然不是合理的职业规划。一些经济学家认为,家庭福利政策的匮乏实际上反映了美国经济中企业和员工之间权利不平衡的状态。经济产出流入公司利润的份额在飙升,而员工的薪酬却停滞不前,这必然会导致美国陷入人才流失的困境。}\\
		C. 揭示美国人才流失的深层次原因
		\tcblower%
		\texttt{文章中提到'这必然会导致美国陷入人才流失的困境', 但这里并不能够选择原因类选项, 因为其本身并非已经引发的社会现象, 于是无法归因}
	\end{tcolorbox}
\end{quote}

\paragraph{历程类} 文段一般有清晰时间轴,介绍整个发展历程

\paragraph{具体类} 可归纳概括出具体小点 ''12345'' (类似申论概括题目) 或展开介绍其中的''1''

\paragraph{比较类} 文段有双主语,双主语地位相当,论述篇幅也大致相当

\paragraph{宽泛+具体} 选项中宽泛和具体的选项同时出现时选择具体的选项

\begin{quote}
	\begin{tcolorbox}[colback=blue!5, colframe=blue!75!black, title=宽泛+具体案例]
		\textit{教材是用来教育的,什么样的教材决定了孩子们将接受什么样的教育。这是公众关注课改的实质。然而长期以来,在教材编写方面,一直缺乏畅通公众意见的制度管道。这是因为我国实行的是一种编、审、用一体化的集权型体制,教材编写机构编什么,学生就得接受什么,即使是对教材的审定,也仍是一种体内循环机制。这种教材编写模式的结果是,我们只可能有一本教材,学校没有其他选择的可能。显然,在一个有着越来越多公众表达的时代,这种自闭式教材编写模式,没得到重视与警惕。}\\
		这段文字意在表明:\\
		A.教材编写应听取公众意见\\
		B.教材的编写模式亟需改变\\
		C.自闭式教材编写不符合时代发展\\
		D.教材编写直接影响孩子教育
		\tcblower%
		\texttt{选项 B, C 和 D 都是宽泛的表述, 即指出存在的问题而没有点出具体的解决方案, 故指出解决方案的选项 A 正确}
	\end{tcolorbox}
\end{quote}