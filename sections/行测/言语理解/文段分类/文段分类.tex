\subsection{文段分类}

\subsubsection{文段类型分类}

\paragraph{说理文}

说理文是以说理为主要目的的文章,通常存在着号召、劝说、启发、警示等目的。在选择主旨大意时\textbf{优先选择阐明道理}的选项。

\paragraph{科普文}

科普文是以传播科学知识为主要目的的文章,通常存在着介绍、说明、分析等目的。在选择主旨大意时\textbf{优先选择介绍事物原理}的选项。

\paragraph{新闻文}

新闻文是以传播新鲜事物为主要目的的文章,通常存在着介绍、报道、评论等目的。在选择主旨大意时\textbf{优先选择表述新突破新进展}的选项。

\subsubsection{文段对策分类}

\paragraph{祈使建议}

文段中包含''应该''、''必须''、''要求''、''可以''、''建议''等词语时, 应当优先选择''建议''、''对策''、''措施''等选项。

\paragraph{禁止否定}

文段中包含''不允许''、''禁止''、''不可以''、''避免''等词语时, 应当优先选择''禁止''、''否定''等选项。

\paragraph{逻辑关联}

文段中包含''如果$\dots$才''、''只有$\dots$就''、''必须$\dots$否则''等词语时, 应当优先选择''因果关系''、''条件关系''等选项。

\paragraph{必要条件}

文段中包含``必须``、``必要``、``不可或缺``等词语时, 应当优先选择``必要条件``、``前提条件``等选项。

\paragraph{语义强调}

文段中包含''时代命题''、''必经之路''等词语时, 应当优先选择''强调''、''特别注意''等选项。

\subsubsection{文段结构分类}
\paragraph{总分结构} 总分结构优先选总结句, 其次是所有分论点的概括
\begin{multicols}{2}
	\begin{enumerate}
		\item 此外 + 分
		\item 这些 + 总
	\end{enumerate}
\end{multicols}

\paragraph{并列结构} 答案要涵盖文段并列的所有内容
\paragraph{转折结构} 转折结构优先选择转折前+后的内容, 其次是转折前后的内容
\paragraph{对比结构} 对比结构优先选择对比前+后的内容, 其次是对比前后的内容
\paragraph{因果结构} 因果结构优先选择 xxx 的原因选项
\paragraph{背景+内容} 背景+内容优先选择同内容表述一致的选项

\begin{quote}
	\textbf{背景知识也需要考虑}
	\begin{tcolorbox}[colback=blue!5, colframe=blue!75!black, title=背景+内容案例]
		\textit{传统的文明标准有三个,即文字、青铜器和城市。但后来许多考古学家发现,由于区域不同,文明的
			差异也很大,这个判断标准不但行不通,还会给考古工作带来阻碍。现如今,国际上对文明的判断标准主
			要是:已进入国家社会形态,具有一套礼仪系统和统治管理制度。良渚文化虽未有青铜器和较为成熟的文
			字,但具有早期国家社会形态和较为完整的礼仪系统以及管理制度,还具有许多东方文明的因素。有良渚
			文化考古研究员甚至认为,良渚文化是东方早期一个集大成的文明。}
		\tcblower%ww
		只看内容部分`\textit{虽未有青铜器和较为成熟的文字,但具有早期国家社会形态}`可以得出答案应该表述为 xx 颠覆了通常标准的选项. 但是要看到背景处`\textit{已进入国家社会形态,具有一套礼仪系统和统治管理制度}`的表述则需要选择满足通行文明判断标准的选项
	\end{tcolorbox}
\end{quote}

\paragraph{设问+内容} 设问+内容优先选择同回答设问的内容表述一致的选项
