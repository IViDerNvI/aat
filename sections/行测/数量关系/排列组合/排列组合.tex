\subsection{排列组合}

\subsubsection{排列组合的基本概念}

\paragraph{排列} 指从一组元素中选出若干个元素进行有序的排列。对于 $n$ 个元素中选出 $r$ 个元素进行排列,记作 $A_n^r$,其计算公式为:

\[
    A_n^r = \frac{n!}{(n-r)!}
\]

\paragraph{组合} 指从一组元素中选出若干个元素进行无序的组合。对于 $n$ 个元素中选出 $r$ 个元素进行组合,记作 $C_n^r$,其计算公式为:

\[
    C_n^r = \frac{n!}{r!(n-r)!}
\]

\subsubsection{排列组合的应用}

\paragraph{插板法} 用于解决将 $n$ 个相同的物品放入 $k$ 个不同的盒子中的问题。可以将问题转化为在 $n+k-1$ 个位置中选择 $k-1$ 个位置放置分隔符。

\begin{quote}
    \begin{tcolorbox}[colback=red!5!white, colframe=red!75!black, title=插板法]
        \textit{将 5 个相同的球放入 3 个不同的盒子中,问有多少种不同的方案}
        \tcblower%
        \begin{center}
            \begin{tikzpicture}
                \foreach \x in {1, 2, 3, 4, 5} {
                        \node[circle, draw, fill=blue!30] at (\x, 0) {};
                    }
                \foreach \x in {6, 7} {
                        \node[rectangle, draw, fill=red!30] at (\x, 0) {};
                    }
                \node at (1, -0.5) {球};
                \node at (6, -0.5) {分隔符};
            \end{tikzpicture}
        \end{center}
        \texttt{将 5 个球和 2 个分隔符(插板)放在一起,共有 $5 + 2 = 7$ 个位置,从中选择 2 个位置放置分隔符,方案数为 $C_7^2 = 21$}
    \end{tcolorbox}
\end{quote}

\paragraph{捆绑法} 对于可以看作一个整体的元素进行组合时,可以将其视为一个单独的元素进行排列或组合。

\begin{quote}
    \begin{tcolorbox}[colback=red!5!white, colframe=red!75!black, title=捆绑法]
        \textit{将 3 个白球和 2 个黑球放在一起,要求黑球互相临近,问有多少种不同的排列方式}
        \tcblower%
        \begin{center}
            \begin{tikzpicture}
                % White balls
                \foreach \x in {1, 2, 3} {
                        \node[circle, draw, fill=white] at (\x, 0) {};
                    }
                % Black balls (bundled)
                \node[rectangle, draw, fill=black!30, minimum width=1.5cm, minimum height=0.8cm] at (4.5, 0) {};
                \node[circle, draw, fill=black] at (4, 0) {};
                \node[circle, draw, fill=black] at (5, 0) {};
                % Labels
                \node at (1, -0.5) {白球};
                \node at (4.5, -0.7) {捆绑的黑球};
            \end{tikzpicture}
        \end{center}
        \texttt{将 2 个黑球视为一个整体(捆绑),
            则有 3 个白球和 1 个大黑球进行排列,共有 $A_4^1 = 4$ 种排列方式。}
    \end{tcolorbox}
\end{quote}

\paragraph{插空法} 要求部分元素不许相邻时,可以先将其他元素排好,再将特殊元素插入在空隙中。注意与插板法的区别。

\begin{quote}
    \begin{tcolorbox}[colback=red!5!white, colframe=red!75!black, title=插空法]
        \textit{将 3 个白球和 2 个黑球放在一起,要求黑球不许相邻,问有多少种不同的排列方式}
        \tcblower%
        \begin{center}
            \begin{tikzpicture}
                % White balls
                \foreach \x in {1, 3, 5} {
                        \node[circle, draw, fill=white] at (\x, 0) {};
                    }
                % Gaps (vertical lines)
                \foreach \x in {0, 2, 4, 6} {
                        \draw[thick] (\x, -0.5) -- (\x, 0.5);
                    }
                % Labels
                \node at (0, -1) {空隙};
                \node at (1, -1) {白球};
                \node at (2, -1) {空隙};
                \node at (3, -1) {白球};
                \node at (4, -1) {空隙};
                \node at (5, -1) {白球};
                \node at (6, -1) {空隙};
            \end{tikzpicture}
        \end{center}
        \texttt{先将 3 个白球排好,形成 4 个空隙(前、后和两个中间),然后在这些空隙中选择 2 个放置黑球。共用 $C_3^3C_4^2$}
    \end{tcolorbox}
\end{quote}

