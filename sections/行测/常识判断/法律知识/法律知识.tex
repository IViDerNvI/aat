\subsection{法律知识}

\subsubsection{国家的基本制度}

\paragraph{人民民主专政制度} 人民民主专政是我国的国体,其主要特色有中国共产党领导的多党合作和政治协商制度、爱国统一战线。

\paragraph{人民代表大会制度} 我国政权的组织形式,政体,根本政治制度,其基本内容为:

\begin{itemize}
	\item 国家的一切权利属于人民
	\item 人民在民主基础上选派代表,组成全国人大和地方人大作为权力机关
	\item 国家检察机关、监察机关、审判机关、行政机关由人大产生,对它负责,受它监督
	\item 人民代表大会常务委员会对本级人民代表大会负责,人民代表大会对人民负责
\end{itemize}

\paragraph{基本经济制度} 在社会主义初级阶段,坚持公有制为主体、多种所有制经济共同发展,坚持按劳分配为主体、多种分配方式并存。国家实行社会主义市场经济。社会主义公有制是我国经济制度的基础,非公有制经济是社会主义市场经济的重要组成部分。

\paragraph{选举制度} 其原则包括普遍性原则,平等原则,直接选举和间接选举并用原则,秘密投票原则。

\paragraph{特别行政区制度} 特别行政区享有高度自治权,包括行政管理权,立法权,独立的司法权和终审权,自行处理有关对外事务的权力。除外交和国防事务外,中央政府不干预特别行政区的内部事务。全国人大具有备案审查权,特别行政区制定的基本法须在全国人大备案,备案不影响其生效,全国人大可发回其中条款,发回的条款除另有规定外,立即失效且无溯及力。

\paragraph{民族区域自治制度} 在国家统一领导下,各少数民族聚居的地方实行区域自治,设立自治机关,行使自治权的制度。

\paragraph{基层群众自治制度} 以中国共产党基层组织为领导,依托基层群众自治组织(例如村委会、居委会),在城乡地区实现居民直接行使相关政治权利及所在居民的自我管理、自我服务、自我教育、自我监督的制度。

\subsubsection{公民的基本权利和义务}

\paragraph{公民的基本权利} 公民的基本权利包括平等权,政治权利和自由,监督权和获得赔偿权,宗教信仰自由,社会经济权利,文化教育权利,特定人的权利

\paragraph{公民的基本义务} 服兵役,纳税等

\subsubsection{国家机构}

\paragraph{国家机构的组织和活动原则}

\subparagraph{民主集中制} 民主基础上的集中和集中指导下的民主相结合

\subparagraph{社会主义法制原则} 国家机关和工作人员行使权力必须在法律框架内

\subparagraph{责任制原则} 国家机关和工作人员必须对其行使权力的后果负责

\subparagraph{为人民服务原则} 国家机关和工作人员接受人民监督, 为人民服务

\subparagraph{精简和效率原则} 提高工作效率, 反对官僚主义

\paragraph{全国人民代表大会} 全国人民代表大会是最高国家权力机关,地方各级人民代表大会是地方国家权力机关。

\paragraph{全国人民代表大会常务委员会} 全国人民代表大会常务委员会是全国人民代表大会的常设机关,行使全国人大闭会期间的部分职权。

\paragraph{国家主席} 中华人民共和国主席(简称“国家主席”),是中华人民共和国的国家代表,也是国家机构之一。中华人民共和国主席、副主席由全国人民代表大会选举产生,每届任期同全国人民代表大会每届任期相同。

\paragraph{国务院} 中央人民政府,是最高国家权力机关的执行机关,是最高国家行政机关。国务院由总理、副总理、国务委员、各部部长、各委员会主任、中国人民银行行长、审计长、秘书长组成。国务院实行总理负责制。总理领导国务院的工作。 [1]

\paragraph{中央军事委员会} 直接领导全国武装力量。其组成人员由中国共产党中央委员会决定。

\paragraph{监察委员会} 行使国家监察职能的专责机关,对所有行使公权力的公职人员进行监察,调查职务违法和职务犯罪,开展廉政建设和反腐败工作,维护宪法和法律的尊严。

