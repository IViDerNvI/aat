\section{附录:逻辑推理常见词语及关系}
\label{trd:appendix}

\subsection{常见词语及其释义}

\begin{longtable}{|p{0.1\textwidth}|p{0.35\textwidth}|p{0.1\textwidth}|p{0.35\textwidth}|}
    \hline
    \textbf{词语} & \textbf{释义}      & \textbf{词语} & \textbf{释义}        \\
    \hline
    珍珠婚         & 像珍珠一样的婚姻,指结婚30周年 & 面包树         & 果实味道像面包的树,原产于太平洋地区 \\
    \hline
    香蕉水         & 味道像香蕉的水,通常       & 玻璃水         & 挡风玻璃清洁液,也称雨刷液、风挡液  \\
    \hline
    缁衣          & 黑色的衣服            & 赤袍          & 红色的衣服              \\
    \hline
\end{longtable}

\subsection{常见词语及其关系}

\begin{longtable}{|p{0.1\textwidth}|p{0.35\textwidth}|p{0.1\textwidth}|p{0.35\textwidth}|}
    \hline
    \textbf{词语} & \textbf{释义}           & \textbf{词语} & \textbf{释义}           \\
    \hline
    墨鱼,乌贼       & 墨鱼和乌贼是同一种动物的不同名称      & 国有企业,集体企业   & 二者从属于内资企业,互相并列        \\
    \hline
    南京,金陵       & 南京是金陵的别称,二者是同一地名的不同称呼 & 粽子,香黍       & 粽子是香黍的别称,二者是同一食物的不同称呼 \\
    \hline
    扬州,广陵       & 扬州是广陵的别称,二者是同一地名的不同称呼 & 荷花,菡萏       & 荷花是菡萏的别称,二者是同一植物的不同称呼 \\
    \hline
\end{longtable}

\subsection{常见古语及其考法}

\begin{longtable}{|p{0.1\textwidth}|p{0.35\textwidth}|p{0.1\textwidth}|p{0.35\textwidth}|}
    \hline
    \textbf{词语}     & \textbf{释义} & \textbf{词语}     & \textbf{释义} \\
    \hline
    己所不欲,勿施于人       & 推广关系        & 不入虎穴,焉得虎子       & 充分条件关系      \\
    \hline
    不忘初心,方得始终       & 因果关系        & 要想人不知,除非己莫为     & 充分条件关系      \\
    \hline
    少壮不努力,老大徒伤悲     & 因果关系        & 不识庐山真面目,只缘身在此山中 & 因果关系        \\
    \hline
    纸上觉来终觉浅,绝知此事要躬行 & 因果关系        & 商女不知亡国恨,隔江犹唱后庭花 & 因果关系        \\
    \hline

\end{longtable}