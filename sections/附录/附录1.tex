\section{附录:成语释义}
\label{fst:appendix}

\begin{longtable}{|p{0.1\textwidth}|p{0.35\textwidth}|p{0.1\textwidth}|p{0.35\textwidth}|}
    \hline
    \textbf{成语} & \textbf{释义}                        & \textbf{成语} & \textbf{释义}               \\
    \hline
    英勇无畏        & 英勇无畏                               & 前仆后继        & 前人倒下了, 后人继续前进             \\
    \hline
    视死如归        & 视死如归, 不怕牺牲                         & 大义凛然        & 正义的样子                     \\
    \hline
    挺身而出        & 勇敢地站出来                             & 勇往直前        & 勇敢地向前走                    \\
    \hline
    踔厉奋发        & 振奋精神, 努力向前                         & 笃行不怠        & 坚定地实践, 不懈怠                \\
    \hline
    赓续前行        & 继续前进                               & 奋楫争先        & 争先恐后, 努力向前                \\
    \hline
    砥砺前行        & 克服困难, 坚持前进                         & 勇毅前行        & 勇敢坚定地向前                   \\
    \hline
    一鼓作气        & 一口气完成, 不松懈                         &             &                           \\
    \hline
    偏听偏信        & 只听一方                               & 亦步亦趋        & 强调跟着别人做                   \\
    \hline
    人云亦云        & 强调跟着别人说                            & 盲从盲信        & 强调不加自考听别人                 \\
    \hline
    玉树临风        & 形容男子英俊潇洒                           & 国色天香        & 形容女子美丽动人                  \\
    \hline
    面如冠玉        & 形容男子面容清秀                           & 螓首峨眉        & 形容女子眉清目秀                  \\
    \hline
    倾国倾城        & 形容女子美丽动人                           & 明眸皓齿        & 形容女性眼眸明亮                  \\
    \hline
    潜移默化        & 环境中受到影响                            & 耳濡目染        & 听看中受到影响                   \\
    \hline
    如沐春风        & 强调打造好环境                            & 润物无声        & 文化教育                      \\
    \hline
    成风化人        & 形成一种风气感化人                          & 耳提面命        & 严格教导                      \\
    \hline
    循循善诱        & 苦口婆心劝导                             & 诲人不倦        & 教育不知疲倦                    \\
    \hline
    春风化雨        & 不强迫教育                              & 和风细雨        & 批评劝导                      \\
    \hline
    振聋发聩        & 唤醒麻木的人                             & 醍醐灌顶        & 一下明白                      \\
    \hline
    颠扑不破        & 比喻真理经得起考验                          & 屡试不爽        & 多次尝试(某种方法)都没有差错           \\
    \hline
    慎终追远        & 谨慎地对待父母的去世,追念祖先的德行                 & 光耀门楣        & 使家族或家庭因某人的成就而荣耀显赫         \\
    \hline
    浑水摸鱼        & 借混乱之机谋取私利                          & 趁火打劫        & 在他人危难时趁机图谋利益              \\
    \hline
    矜功伐善        & 夸耀自己的功劳与优点                         & 色厉内荏        & 外表强硬,内心虚弱                 \\
    \hline
    沸反盈天        & 声音喧闹如沸腾的水,响彻天空。形容极度喧嚣混乱            &             &                           \\
    \hline
    司空见惯        & 常见,强调不足为奇                          & 层见叠出        & 常见                        \\
    \hline
    形如槁木        & 瘦的像木头                              & 江心补漏        & 形容来不及补救                   \\
    \hline
    鲁鱼亥豕        & 一字之差,差之千里                          &             &                           \\
    \hline
    千锤百炼        & 原指金属要经过多次锤打才能成材,后比喻人在艰难困苦中不断磨练、成长。 & 精雕细琢        & 原指雕刻细致入微,后引申为做事认真细致、反复打磨。 \\
    \hline
    精益求精        & 已经很好了,还要求更好,形容追求极致、不断完善。           &             &                           \\
    \hline
\end{longtable}

